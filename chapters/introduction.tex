%!TEX root = ../dissertation.tex

\chapter{Introduction}
\label{chapter:introduction}

% IoT and LPWAN/LoRa --> disaster scenarios/monitoring --> Drones --> Drones Supporting IoT in disasters

The ubiquity of the Internet, new and innovative communication protocols and the miniaturization of computational devices gave rise to a new paradigm called \gls{IoT}. The number of \gls{IoT} devices is projected to grow steadily in the years to come \footnote{https://www.statista.com/statistics/471264/iot-number-of-connected-devices-worldwide/}, especially thanks to their possible applications in a vast and diverse range of fields. \gls{IoT} solutions are in fact being used in the health care, in the industry, in agriculture, in cities and in many other sectors \cite{ref:iot-apps}. In many cases, \gls{IoT} devices are battery powered and subject to very strict power constraints. For this reason, a new range of low power wireless communication protocols have been developed and standardized in order to support the operation of \glspl{LPWAN}. These networks are typically formed by inexpensive, simple devices that need to communicate infrequently over long distances at low bitrates. LoRa, acronym that stands for Long Range, is one of the most promising and versatile technologies enabling \glspl{LPWAN}. LoRa takes advantage of \gls{SS}, codes orthogonality and the good propagation characteristics of the sub-GHz spectrum to provide reliable communication over long distances at the expense of the bit rate and of the transmission frequency of packets due to duty cycle limitations in the bands typically used by the protocol. \\
\glspl{UAV}, commonly called drones, are flying vehicles that can operate without the need of a remote pilot. \glspl{UAV} have been confined to military applications for a long time, but, thanks to the drop of the price of the components, \glspl{UAV} are now commercialized and used in a wide range of civil applications. \glspl{UAV} usage is increasingly being considered in operations taking place in harsh, inaccessible or dangerous environments, where human intervention is limited or impossible. In case of a disaster, such as an earthquake, a flood or a wildfire, \glspl{UAV} can be used to support the operations of the rescuers, to localize the victims and to create a backhaul network in case of absence or disruption of the communication facilities. In these last circumstance, the relay network formed by \glspl{UAV} can support the data generated by isolated \gls{IoT} devices that have been deployed in advance or during the rescue operations. 

\section{Motivation}
\label{sec:motivation}

The integration of \glspl{UAV} and \gls{LPWAN} protocols in disaster scenarios may offer a new cost-effective and energy efficient way of tackling many problems arising during the operations of the rescuers. Unfortunately, the related literature is still scarce, therefore leaving many issues open to further investigation. In this work, a particularly challenging scenario is considered: a forest in which a wildfire has been detected.
Wildfires represent a worldwide problem, but America and Europe are the two continents where the incidence of forest fires is higher. The costs, in terms of economic losses and human deaths, are considerable \cite{ref:disasters-report} \cite{ref:wildfires-figures}.
Wildfires typically affect rural or suburban areas where the network coverage of traditional communication networks (e.g. cellular networks) is scarce, intermittent or completely lacking. This situation is worsened by the huge propagation loss introduced by foliage and trees in the signal propagation path. In this situation \glspl{UAV} can be used to establish a mesh network acting as a relay between the \gls{CP} managing the operations and the sensors carried by firemen, therefore providing more precise situational awareness to the rescuers. The mobility of the firemen represents another huge challenge, since the \glspl{UAV} mesh network needs to adjust its position while maintaining the connectivity to the \gls{CP} and to firemen on the ground. 
Thanks to the good properties
In this thesis, a solution based on LoRa and WiFi is investigated. Thanks to its good propagation characteristics, its power efficiency and its versatility, LoRa represents a good candidate for sending data generated by firemen sensors. A LoRa chip is carried by firemen and takes care of periodically transmitting relevant data. The \glspl{UAV} are equipped with LoRa gateways able to receive data from the ground LoRa devices and relaying that data over WiFi to the \gls{CP}. WiFi is a widely used protocol that can provide high data rates in the unlicensed 2.4 GHz bandwidth, thus reducing the formation of bottlenecks in the mesh network.

\section{Objectives}

The main objective of the present work is to evaluate the performances of the architecture described in section \ref{sec:motivation} through simulations performed using \gls{ns-3}. A suitable algorithm to form the WiFi mesh network is also implemented.
The objectives are enumerated below:
\begin{itemize}
	\item Implement in a simulation the LoRa-WiFi network architecture and evaluate its performance in terms of selected QoS parameters;
	\item Properly characterize the propagation channel of \gls{UAV} to \gls{UAV} communication and ground nodes to \gls{UAV} communication taking into account the presence of trees in the propagation path; 
	\item  Implement one or more suitable \glspl{UAV}' mobility models based on centralized and/or distributed algorithms for the formation of a connected and dynamic mesh network;
	\item Define a suitable and lightweight protocol for delivering commands to the \glspl{UAV} and exchanging messages between \glspl{UAV}  
\end{itemize}

\section{Thesis Outline}




%A demonstration of how to use acronyms and glossary:
%
%A \gls{MSc} entry.
%
%Second use: \gls{IST}.
%
%Plurals: \glspl{MSc}.
%
%A citation example \cite{nobody}
