%!TEX root = ../dissertation.tex

\chapter{Background}
\label{chapter:background}

%\section{The Internet of Things}

\section{Low Power Wide Area Networks}

The number of \gls{IoT} devices is expected to grow significantly in the years to come. The year 2017 will be remembered as the year in which the number of connected devices surpassed the world population \footnote{https://www.statista.com/statistics/764026/number-of-iot-devices-in-use-worldwide/}.
In many applications, such as smart cities, environmental monitoring and smart agriculture, these devices are densely deployed and powered by traditional AA batteries. For these reasons, they must be able to operate under strict power constraints using simple and cheap hardware components. Transmitting data over a wireless medium is a very costly operation in terms of power, due to many unpredictable environmental factors such as heat, humidity, wind and buildings and due to the inherent complexity of traditional wireless technologies.
Protocols like \gls{BLE} and ZigBee try to address the power limitation problem when short-range communication is needed, but for a vast number of IoT applications, where a high number of devices are deployed over large areas (e.g. a city), these short range solutions are suboptimal and expensive due to the complex and dense network infrastructure that needs to be installed. Cellular networks cover large areas, but are not power efficient. Moreover 2G, 3G and LTE connectivities are typically available in urban areas, but not always in suburban or rural areas. To address the aforementioned problems, a new range of protocols and technologies are being developed with the explicit purpose of enabling the operation of \glspl{LPWAN}. As the name suggests, \glspl{LPWAN} have the primary objective of providing low-power and low-cost connectivity over large geographical areas. The applications are multiple and diverse: smart cities, smart grids, smart metering, logistics, industrial monitoring, smart agriculture, etc.
Recently, all the principal \glspl{SDO} such as the \gls{IETF}, the \gls{IEEE}, the \gls{3GPP} and the \gls{ETSI} are intensifying their efforts in the standardization of \gls{LPWAN} protocols. Moreover, numerous consortiums and alliances works to promote specific \gls{LPWAN} technologies. Examples of such alliances are the LoRa Alliance promoting the LoRaWAN protocol, the Dash7 Alliance promoting the Dash7 Alliance protocol and the Wightless-SIG promoting the Weightless protocol. Even if the existing \gls{LPWAN} protocols and solutions are extremely diverse and address different niches in the same application segment, most of them use similar solutions to solve the same range of problems. A quick summary is given below:

\begin{itemize}
\item \textbf{Use of sub-GHz bands:} the vast majority of \gls{LPWAN} protocols take advantage of the good qualities of sub-GHz bands in terms of attenuation, multipath fading and obstacle penetration. Moreover, sub-GHz bands are currently less congested than the overly used 2.4 GHz band. Most \gls{LPWAN} technologies (e.g. LoRa and Sigfox) use unlicensed \gls{ISM} bands, while cellular operators are trying to exploit already owned licensed frequency bands. The use of unlicensed \gls{ISM} bands reduces the operational costs of running the network, but with the drawback of having to compel with the rules mandated by legislative authorities for the usage of these bands. As an example, in Europe, the 868.0-868.6 MHz band is subject to a 1\% duty cycle limitation \cite{missing}, meaning that, in a day, a single device can transmit for at most 14.4 minutes.

\item \textbf{Use of \gls{NB} or \gls{SS} modulations:}\gls{NB} modulations have the advantage of reducing the adverse effects of noise on the transmitted signals. Using these techniques, receivers can successfully recover severely attenuated signals and achieve a sensitivity level of even -130 dBm \cite{missing}. On the other side, the small bandwidth only allows low data rates in the order of few kbps or even a few hundred of bps if an \gls{UNB} modulation is used. In this case the bandwidth can be as low as 100 Hz. Some protocols, such as LoRa, use variations of \gls{SS} techniques to spread the narrowband signal over a wider bandwidth. The resulting signal has noise-like characteristics that makes it more resistant to jamming and eavesdropping and more resilient to interference. To overcome the less efficient use of bandwidth, \gls{SS} is typically used in conjunction with orthogonal codes. This allows multiple overlapping signals that are spread using different codes to be successfully recovered at the receiver.

\item \textbf{Low power operation:} in a \gls{LPWAN}, end devices are typically battery-powered and might need to operate using the same battery for many years. To reduce power consumption, \gls{LPWAN} networks are organized in a simple and efficient star topology or a star-of-stars topology, where all the devices are directly connect to one or multiple gateways or base stations. Multi-hop protocols are typically not used since some network nodes, called hot-spots, might experience more traffic than the others. Moreover, hot-spots have to listen periodically for new messages coming from the other nodes thus reducing their overall battery lifetime. Duty-cycling the devices, i.e. alternating ON-OFF communication periods, is another frequently used technique to reduce the power consumption.
For example, an uplink transmission can start only when new data is ready and a downlink reception can start only at a predefined scheduled time, usually following an uplink transmission. Simple random access ALOHA-like MAC protocols, requiring simple and cheap hardware and no synchronization, are typically used. Finally, to save on computational power, some complex operations, like detecting duplicated packets, are offloaded to the backend network.

\item \textbf{Low cost deployment:} installing, maintaining and operating an \gls{LPWAN} network must be as cheap as possible. \glspl{LPWAN} can achieve a chip cost as low as 1-2 \$ with an annual operation cost of 1\$. This can be achieved by resorting to simple and sometimes inaccurate hardware components. The network backbone usually requires only few base stations to cover areas of tens of kilometres, thus additionally reducing the deployment cost of the network. Finally, most \gls{LPWAN} solutions use unlicensed \gls{ISM} bands thus avoiding wasting money for the acquisition of licenses.

\item \textbf{Scalability:} support of a large number of devices is usually achieved by diversifying as much as possible in time, space and channels. Since devices use simple hardware, this diversification is usually implemented in the backend by using multi-channel multi-antenna base stations to parallelise the reception of signals. Some protocols also implement mechanisms to dynamically adapt the data rate and to select the best channel, but given the strong link asymmetry of most \gls{LPWAN} solutions, this mechanism, highly reliant on downlink information transmitted by the base station, is quite limited.

\end{itemize}

\subsection{LPWAN Technologies}

\subsection{LoRa and LoRaWAN}

\subsection{Comparison of LPWAN protocols}

\section{Unmanned Aerial Vehicles}

\subsection{UAV-assisted disaster management}